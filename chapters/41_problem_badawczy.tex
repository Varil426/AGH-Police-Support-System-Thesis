\section{Problem badawczy}
\label{sec:problemBadawczy}

\par Zaprojektowany system pozwala śledzić i symulować działania jednostek policji w inteligentnym mieście. Głównym celem jest zbadanie, jak w efektywny sposób można zarządzać tymi jednostkami. Sprawdzenie wpływu kontekstu podczas ich rozmieszczania, jak i w trakcie wydawania poleceń.

\par System posiada wiele zmiennych, które umożliwiają jego konfigurację w zależności od potrzeb. Istotnym jest jednak wybranie konkretnych cech, którymi chcemy manipulować, aby otrzymać pełen obraz jego działania.

\par Aby dobrze ocenić otrzymane rezultaty, koniecznym jest wybranie cech, które zostaną porównane. Więcej na ten temat zostało opisane w podrozdziale \ref{sec:wyborMetodyBadawczejIUzasadnienie}.