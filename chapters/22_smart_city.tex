\section{Smart City - Inteligentne miasto}

\par Termin "inteligentnego miasta"\english{Smart City} po raz pierwszy pojawił się w styczniu 2000 roku, kiedy \emph{Robert E. Hall} (TODO Dodać przypis) opublikował artykuł zatytułowany "The Vision of A Smart City". Nazwa ta zyskała na popularności między latami 2005 a 2008, stając się szeroko wykorzystywana przez potężne firmy technologiczne, takie jak IBM i Cisco.

\par Koncepcja ta opiera się na rozwoju miast, które wykorzystują zaawansowane technologie informatyczne, komunikacyjne i inne innowacyjne rozwiązania w celu poprawy jakości życia mieszkańców oraz efektywności zarządzania miastem. Kluczowym celem Smart City jest tworzenie bardziej zrównoważonych, ekologicznych i efektywnych miast, które mogą skuteczniej radzić sobie z rosnącymi wyzwaniami urbanizacji.

% TODO Dodać przypis
\par Kluczowymi elementami są:
\begin{itemize}
    \item Inteligentna infrastruktura\english{Smart Infrastructure} - wykorzystanie zaawansowanych technologii w celu doskonalenia infrastruktury miejskiej i regionalnej. Wykorzystywane są rozwiązania takie jak internet rzeczy\english{Internet of Things}, analiza danych, sztuczna inteligencja i inne innowacyjne technologie, które pozwalają na zwiększenie efektywności, promowanie zrównoważonego rozwoju oraz podniesienie jakości świadczonych usług w sektorach takich jak transport, energetyka, wodociągi i wiele innych. Głównym celem inteligentnej infrastruktury jest zapewnienie mieszkańcom wyższej jakości życia oraz efektywnego zarządzania zasobami przy minimalnym wpływie na środowisko.
    \item Inteligentne Środowisko i Zrównoważalność\english{Smart Environment and Sustainability} - tworzenie bardziej zrównoważonych, ekologicznych i przyjaznych dla ludzi miejsc, które pozwalają na długotrwały rozwój przy minimalnym negatywnym wpływie na środowisko.
    \item Open Data i łączność wzajemna\english{Open Data and Interconnectivity} - otwartość dla ludzi, firm i rządów celem zapewnienia nowych serwisów, współpracy i innowacji.
    \item Inteligentna mobilność i transport\english{Smart Mobility and Transportation} - wpływ na transport i komunikację ludzi celem zmniejszenia obciążenia ulic.
    \item Inteligentny obywatel\english{Smart Citizen} - ludzie są najbardziej istotnym elementem inteligentnego miasta, ponieważ to oni je zamieszkują. Główną ideą jest poprawa ich życia - ułatwienie go, zautomatyzowanie i zoptymalizowanie.
    \item Inteligentna technologia\english{Smart Technology} - oznacza technologię w sercu \emph{Smart City}. Urządzenie te zbierają informację na podstawie licznych sensorów, co pozwala im dowiedzieć się o otaczającym ich świecie. Przykładem takiego urządzenia, może być posiadany niemal przez każdego \emph{Smartphone}, potrafiący zbierać wiele istotnych informacji.
\end{itemize}

\par Oczywiście, przedstawione tutaj elementy nie wyczerpują wszystkich aspektów inteligentnego miasta, gdyż istnieje wiele różnych definicji tego terminu. Niemniej jednak, to właśnie te elementy pojawiały się najczęściej w literaturze i debatach na ten temat. Wszystkie te składniki tworzą kompleksowy ekosystem, w którym wiele elementów jest ze sobą powiązanych i współpracuje, dążąc do poprawy jakości życia i ułatwienia codziennego funkcjonowania wszystkich mieszkańców \emph{Smart City}.


