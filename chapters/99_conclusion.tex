\chapter{Podsumowanie i wnioski}

\par Aby spełnić założenia projektu, została przeprowadzona analiza systemów agentowych, systemów rozproszonych, komunikacji w ramach systemów rozproszonych, jak i wykorzystania map \emph{OSM}. Zbadane zostały również dane kontekstowe, pojawiające się w systemie w odniesieniu do obcujących z nimi agentów. Dane te zostały skategoryzowane, jak i została pokazana ich rola w systemie.

\par W ramach tej pracy, został zaimplementowany rozproszony system mikroserwisowy, z serwisami, w których działają agenci. Dodatkowo została również zaimplementowana symulacja, pozwalająca na przetestowanie działania systemu i przeprowadzenie na nim eksperymentów. Powstała również aplikacja \emph{web}owa, umożliwiająca wizualizację i agregacją danych, dotyczących przebiegu działania. Całość powstałego rozwiązania łączy się za pomocą \emph{RabbitMQ}. Elementy systemu zostały zintegrowane z \emph{PostGIS}, \emph{pgRouting} i \emph{Grafana Loki}. Sumarycznie, powstała implementacja zawiera ponad $20000$ linijek kodu.

\par Dodatkowo, system został skonteneryzowany, wykorzystując technologię \emph{Docker}. Dzięki temu możliwym jest skalowanie systemu, w przypadku chęci przeprowadzenie eksperymentów, z wykorzystaniem jego bardziej złożonej wersji, w której bierze udział więcej agentów.

\par Stworzony system jest wrażliwy na dane kontekstowe. Informacje te, są wykorzystywane przez agentów, którzy postrzegają i wpływają na swoje otocznie. Jak wykazała analiza, kontekst może pełnić istotną rolę w procesie decyzyjnym i odpowiednia jego interpretacja, może wpłynąć pozytywnie na rezultaty działania systemu. Pozwala on na podjęcie rozsądniejszej decyzji, która przynosi efektywne rezultaty. Nie należy jednak zapominać, że otrzymane wyniki dotyczą ściśle określonej konfiguracji. Powstały system jest jednak narzędziem, pozwalającym na dalsze badania tej dziedziny.