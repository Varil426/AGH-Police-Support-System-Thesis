\section{Problem badawczy i teza}

\par Celem pracy jest zbadanie, czy oraz w jaki sposób algorytmy decyzyjne mogą wpłynąć na efektywność działania jednostek służb porządkowych w inteligentnym mieście. W obecnych czasach, coraz większy wpływ na nasze życia wywierają inteligentne systemy komputerowe, zdolne do podejmowania decyzji. Naszym celem jest zapewnienie, że te decyzje są jak najbardziej rozsądne i prowadzą do osiągnięcia najlepszych możliwych rezultatów. Wyniki te można jasno zdefiniować i porównać ze sobą, w celu wyłonienia najbardziej efektywnych strategii zarządzania dostępnymi zasobami.

\par Bezpiecznym założeniem wydaje się to, że bardziej wyrafinowane algorytmy, które uwzględniają większą liczbę danych wejściowych, takich jak odległości wszystkich patroli od celu, natężenie patroli w danej dzielnicy miasta czy ocena danego rewiru, lepiej poradzą sobie z podejmowaniem decyzji niż te bardziej prymitywne. Jednak pytaniem pozostaje, czy ta różnica jest zauważalna w praktyce.