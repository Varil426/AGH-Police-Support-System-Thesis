\chapter{Wstęp}

\par Współczesna cywilizacja stawia przed sobą liczne wyzwania, które muszą zostać przezwyciężone. W dążeniu do osiągnięcia tego celu nieustannie dąży do optymalizacji i automatyzacji procesów zachodzących w jej obrębie. Począwszy od prostych udogodnień w codziennym życiu, takich jak urządzenia wchodzące w skład inteligentnego domu\english{Smart Home}, aż po systemy zarządzające komunikacją miejską i transportem, a także internetu rzeczy\english{Internet of Things}. Każda dziedzina życia została gruntownie zmieniona dzięki wprowadzeniu komputerów, jednakże to nadal człowiek pozostaje głównym komponentem odpowiedzialnym za podejmowanie decyzji. Oprogramowanie stanowi narzędzie mające na celu wspieranie tego procesu.

\par Niemniej jednak, nasuwa się pytanie, czy jest możliwe przeniesienie części tego obciążenia w kierunku autonomicznego oprogramowania. W szczególności, czy zlecenie maszynom tak krytycznej roli, jaką pełnią służby policyjne, jest korzystnym pomysłem. Warto zauważyć, że wprowadzenie autonomicznego oprogramowania w tak istotnym obszarze wywołuje wiele istotnych kwestii. Problem ten obejmuje jednostki, tj. patrole, które są zobowiązane do funkcjonowania w sposób autonomiczny, jednak równocześnie muszą być zdolne do skutecznej współpracy ze sobą nawzajem oraz do podporządkowania się centralnemu zarządcy.

\section{Zakres pracy}
\label{sec:zakresPracy}

\par Celem tej pracy jest stworzenie rozproszonego, inteligentnego systemu opartego o koncept agentów, wrażliwych zarówno na zmiany w swoim otoczeniu, jak i w innych agentach - czyli na tak zwany kontekst. Tak przygotowany system, zostanie poddany badaniu wpływu danych kontekstowych na jego działanie. W tym celu koniecznym jest istnienie otoczenia, które będzie w stanie reagować na zachowania agentów, jak i generować własne zdarzenia. Tę rolę spełni symulacja, która będzie postrzegana przez system, jako prawdziwy świat.

Planowanym zakresem prac jest:
\begin{itemize}
    \item Przygotowanie standardu komunikacji pomiędzy mikro-serwisami\english{micro-service} w systemie rozproszonym.
    \item Przygotowanie standardu komunikacji między sobą agentów reprezentującymi poszczególnych aktorów w systemie, jak i możliwość postrzegania przez nich otoczenia\english{environment}.
    \item Implementacja systemu rozproszonego, który reprezentuje system policji (jednostki patrolujące, jednostkę centralną).
    \item Implementacja aplikacji webowej pozwalającej na:
    \begin{itemize}
        \item Monitorowanie obecnego stanu systemu.
        \item Generowanie raportów z działania systemu.
    \end{itemize}
    \item Implementacja symulacji, która:
    \begin{itemize}
        \item Symuluje przebieg incydentów.
        \item Symuluje ruch i działania patroli.
        \item Wykorzystuje mapy OpenStreetMap wybranych miast.
    \end{itemize}
    \item Skonteneryzowanie mikro-serwisów.
    \item Implementacja algorytmu decyzyjnego opartego o dane kontekstowe.
    \item Zbadanie wpływu danych kontekstowych na działania systemu.
\end{itemize}

\section{Zawartość pracy}
\label{sec:zawartoscPracy}

Zawartością pracy jest:
\begin{itemize}
    \item Określenie celu i motywacji projektu.
    \item Zbadanie zagadnienia od strony technicznej.
    \item Opisanie zaimplementowanego rozwiązania i zastosowanych technologii.
    \item Prezentacja powstałego rozwiązania.
    \item Przedstawianie zebranych danych oraz sformułowanie wniosków na ich podstawie.
\end{itemize}