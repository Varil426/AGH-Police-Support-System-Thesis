\section{Systemy agentowe}

\par Idea agentów powstała w latach 80. ubiegłego wieku i zyskiwała na popularności w latach 90. Na jej rozwój wpływ miało wielu naukowców i badaczy, w szczególności wyróżniali się:

% TODO Dodać przypisy
\begin{itemize}
    \item Marvin Minsky, autor książki "The Society of Mind" (1986), gdzie przedstawił model ludzkiego umysłu jako kolekcję współpracujących "agentów umysłowych".
    \item John McCarthy, twórca pracy "Programs with Common Sense" (1959), która skupiała się na koncepcji doradztwa bazującego na wiedzy.
    \item  Rodney Brooks, autor artykułu "A Robust Layered Control System for a Mobile Robot" (1986), gdzie przedstawił koncepcję sterowania robotami na bazie serii warstw, z których każda odpowiadała za określony aspekt zachowania.
\end{itemize}

\par Omawiani tutaj agenci mogą działać samodzielnie, ale również współpracować między sobą w celu osiągnięcia wspólnych lub indywidualnych celów. Ich cechami charakterystycznymi są:
\begin{itemize}
    \item Autonomia - oznaczająca możliwość działania bez bezpośredniej kontroli ze strony ludzi. Agent samodzielnie decyduje o podejmowanych działaniach, jak i decyduje o swoim wewnętrznym stanie.
    \item Zdolność do działania - agenci posiadają narzędzia (tak zwane efektory) do wpływania na ich środowisko.
    \item Zdolność do spostrzegania - agenci potrafią zaobserwować zmiany w swoim środowisku za pomocą tak zwanych sensorów.
    \item Celowość - agenci mają określony cel, bądź cele, które próbują spełnić.
    \item Zdolność do komunikacji - określająca możliwość porozumiewania się pomiędzy agentami w systemie.
\end{itemize}

\par Agentów możemy klasyfikować na wiele typów, które wyróżniają się swoim zachowaniem. Podział ten wygląda następująco:
\begin{itemize}
    \item Agent reaktywny\english{Reactive Agent} - działa na podstawie bezpośredniego bodźca ze środowiska. Reaguje na zmieniające się otoczenie. % TODO Dodać przypis Rodney Brooks MIT A Robust Layered Control System for a Mobile Robot
    \item Agent celowy\english{Goal-driven Agent} - posiada określone cele, które próbuje osiągnąć. % TODO Dodać przypis Michaela Wooldridge'a i Nicholasa Jenningsa  "Intelligent Agents: Theory and Practice" (1995)
    \item Agent poznawczy\english{Cognitive Agent} lub agent inteligentny\english{Intelligent Agent} - jest obdarzony zdolnością do "myślenia", planowania i podejmowania decyzji bazując na wiedzy o otoczeniu. % TODO Dodać przypis "Artificial Intelligence: A Modern Approach" Stuarta Russella i Petera Norviga
    \item Agent hybrydowy\english{Hybrid Agent} - są to agenci łączący cechy wyżej wymienionych typów.
\end{itemize}

\par Otoczenie lub środowisko\english{Environment}, które zostało już kilkukrotnie wspomniane, stanowi koncepcję definiującą przestrzeń, w której działa agent. Wydarzenia, które mają miejsce w tym otoczeniu, wpływają na zachowanie agenta, który jest zdolny do podejmowania decyzji i podejmowania działań, które znajdują odzwierciedlenie w tymże środowisku. Otoczenie może przyjmować różne formy, włączając w to fizyczne, wirtualne i symboliczne. Podstawowymi cechami, które określają charakter danego otoczenia, są:
\begin{itemize}
    \item Stan - zestaw cech opisujących dane otoczenie.
    \item Akcje agenta - są to działania, które może podjąć dany agent celem wpłynięcia na środowisko i osiągnięcia postawionych przed nim celów.
    \item Funkcja celu - określa ona jakość akcji agenta. Pozwala określić czy dana akcja będzie miała pozytywny, czy negatywny wpływ na osiągnięcie określonego celu oraz wyłonienie najlepszej decyzji.
    \item Polityka - określa strategię, którą agent wykorzysta podczas podejmowania decyzji w środowisku. Jest ona instrukcją, która definiuje jakie akcje powinien podjąć agent, w odpowiedzi na istniejący lub zmieniony stan środowiska, aby osiągnąć swoje cele.
\end{itemize}